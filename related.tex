% Relationship with Alviano et al. 

%\subsection{}

\def\asplain{ASPLAIN}


The present paper is related closely related to work in the research on xAI in ASP or
Prolog such as the xASP systems \cite{ly:iclp,ly:spie,trieu2022xasp,alviano2024xai} and
the systen  
s(CASP) \cite{AriasCCG20}. 
Since xASP2 is a signficant enhancement of its predecesssors described in
\cite{ly:iclp,ly:spie,trieu2022xasp},  we will discuss the relationship of \asplain{} with
xASP2 and s(CASP). 

xASP2 provides explanations for atoms in an answer set (true atoms) and
atoms not in an answer set (false atoms) by computing a justification for the existence or
(non existence) of an atom in an answer set. 
xASP2 comes with a user interface that allows users to explore the explanations
interactively while s(CASP) does not provide such feature.   
Similar to xASP2, \asplain{} provides a graphical interface that allows users to explore
different traces of a query.  
Furthermore, \asplain{} works with propositional programs, and hence, will need to ground
programs with variables before providing explanations. 
The most significant difference between \asplain{} and xASP2 or s(CASP) is that \asplain{}
focuses on  contrastive explanations. 

Consider a program $P$ and an answer set $M$ of $P$. 
A derivation of $a$ in $MG^M_P$ is a sequence of rules 
$\langle r_1, \ldots, r_n \rangle$ such that 
(\emph{i}) $B^+(r_1) = \emptyset$;   
(\emph{ii}) for $1 \le i\le n$, 
$Lb(r_i)$ is a node of $MG^M_P$ and  
$B^+(r_i) \subseteq \bigcup_{1 \le j< i} (H(r_j) \cap M)$; and
(\emph{iii}) $a \in head(r_n)$. 

It is easy to see that a derivation of $a$ in $MG^M_P = (V,E_+ \cup E_-)$ creates 
a subgraph $(V_a, E_a)$ of $MG^M_P$ 
where  $V_a = Lb(r_i) \cup \bigcup_{1 \le j le n} B^+(r_i) \cup \{a\}$ 
and $E_a = \{(v,v') \mid v, v' \in V_a, (v,v') \in E_+\}$. 
This graph explains why $a$ belongs to $M$ which is similar to an explaination graph
defined in \cite{ly:iclp,ly:spie,trieu2022xasp}. 
Because $(V_a, E_a)$ does not include explanations for negative atoms in $M$ whose non-presence
supports the presence of $a$ in $M$, it corresponds, therefore, to possibly many,
explanation graphs of $a$ according to the definition in
\cite{ly:iclp,ly:spie,trieu2022xasp}. 
Since the notion of explanation derivation in \cite{alviano2024xai} considers only
explanation graphs derived from minimal assumption sets of $M$,  
we can conclude that $(V_a, E_a)$ correponds to potentially many explanation derivations
of $a$ in $M$ as defined in \cite{alviano2024xai}. 


It is worth mentioing that \cite{alviano2024xai} also provides explanations for false
atoms in $M$, i.e., atoms that do not belong to $M$. By design, the present paper provides
explanation for $a \not\in M$ by identifying a contrastive model that supports $a$.  


s(CASP) is a top-down, goal-directed system, and works at the predicate level. 
Built on top of Prolog with ASP features such as constraints or choice atoms, s(CASP)
inherits the advantages of Prolog such as the ability to work with non-grounded atoms,
avoiding the grounding bottleneck of ASP solvers. It searches for an explanation for an
atom by constructing one of its justification trees or a Prolog-style proof trees.
Therefore, s(CASP) does not require an answer set as a part of the input and . 
can provide explanations for atoms with variables.
However, due to its Prolog-based implementation, this process may
return different explanations for the same query when the order of rules or literals in
rules is changed. 

    

% \begin{observation}
% \begin{itemize} 
%   \item every explanation graph of $a$ in $M$ is a subgraph of the model subgraph $MG^M_P$. 
%   \item every explanation graph of $a$ 
%   for every derivation 
% \ned{itemize}   
% \end{observation} 
% Furthermore, $(V_a, E_a)$ represents, possibly many, explanation graphs of $a$ according to the
% definition in \cite{ly:iclp,ly:spie,trieu2022xasp} which is later restricted\footnote{The
% notion of explanation derivation in \cite{alviano2024xai} considers only explanations
% derived from minimal assumption sets of the answer set.} in 
% \cite{alviano2024xai}. 
% Because $(V_a, E_a)$ does not include explanations for negative atoms in $M$ whose non-presence
% supports the presence of $a$ in $M$, it  

% only correponds to the positive part of an
% explanation graph of $a$ in $M$. 

% herefore, we have that there is a one-to-one correspndence between  
% explanation graphs of $a$ in $M$ and  
% derivations of $a$ in $M$, and thus, 
% subgraphs explaining $a$ of the model subgraph $MG^M_P$.
