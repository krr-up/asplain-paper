% Relationship with Alviano et al. 

%\subsection{}

\def\asplain{ASPLAIN}


The present paper is related closely related to work in the research on xAI in ASP or Prolog such as the xASP systems \cite{ly:iclp,ly:spie,trieu2022xasp,alviano2024xai} and the systen  
s(CASP) \cite{AriasCCG20}. 
Since xASP2 is a signficant enhancement of its predecesssors described in \cite{ly:iclp,ly:spie,trieu2022xasp}, we will discuss the relationship of \asplain{} with xASP2 and s(CASP). 
Both xASP2 and s(CASP) provide explanations for atoms in an answer set (true atoms) and
atoms not in an answer set (false atoms) by computing a justification for the existence or
(non existence) of an atom in an answer set. 
xASP2 comes with a user interface that allows users to explore the explanations
interactively while s(CASP) does not provide such feature. 
s(CASP) does not ground the programs before computing explanations and thus, it is not 
affected by the grounding bottleneck of ASP.
However, s(CASP) may provide different explanations towards the same query when the ordering
of rules or literals in rules is changed.

\asplain{} provides a graphical interface that allows users to explore different traces of a query. 
In this sense, it is similar to xASP2 and s(CASP). 
Furthermore, \asplain{} works with propositional programs, and hence, will need to ground
programs with variables before providing explanations. 
The most significant difference between \asplain{} and xASP2 or s(CASP) is that \asplain{} focuses on contrastive explanations.


