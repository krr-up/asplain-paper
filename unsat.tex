\subsection{Explaining Unsatisfiability}

\begin{itemize}
    \item {\color{blue} This definition does not impose the existence of a \emph{reference model} $\m_r \models \pr$}
    This allows finding foils for unsatisfiable reference programs, $AS(\pr) = \emptyset$.
    If $Q = \emptyset$ then we are simply recovering satisfiability.
    Otherwise, if $Q \neq \emptyset$, we are recovering satisfiability and forcing something to be true in the process.
    \item This might not be considered a \emph{contrastive explanation} in the strict sense, as there is no reference model to contrast with.
    \item It is very useful in practice.
    \item We need a different definition for the contrast.
\end{itemize}


\begin{example}{Unsatisfiablity}
    Consider $P3$ as $P2$ extended with the rule:
    \begin{lstlisting}[language=clingos]
    r8 :- not w.
    \end{lstlisting}
    $P3$ is unsatisfiable as there is no answer set that can satisfy rules $r8$,$r7$ and $r3$ at the same time.
    We can compute
    \[
    \bestfoils(P3, \emptyset, \{r8, r7, r3\}, \emptyset, \pdiff)
    \]
    to obtain three foils that make the program satisfiable again by removing one of the conflicting constraints.
\end{example}

\comment{TODO BRAIS: Here we migth want a different definition of contrastive explanation for unsat programs without the reference model.}


